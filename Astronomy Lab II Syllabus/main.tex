\documentclass[11pt]{article}
\usepackage[includeheadfoot, top=1.0in, bottom=1.0in, hmargin=1.0in]{geometry}
\usepackage[utf8]{inputenc}
\usepackage{fancyhdr}
\usepackage{url}
\pagestyle{fancy}
\usepackage{setspace}
\usepackage{tabularx}
\usepackage{hyperref}


\lhead{Astronomy Lab II}
\rhead{Spring 2022}
\lfoot{Golant}
\rfoot{Tues 6-9pm}
\cfoot{\thepage}

\begin{document}
%%%%%%%%%%%%%%%%%%%%%%% INTRO %%%%%%%%%%%%%%%%%%%%%%%
\begin{center}
\LARGE{ASTR1904: Astronomy Lab II -- Section 002} \\ \medskip \Large{Syllabus}\\ 
\end{center}

\noindent
\textbf{Instructor:} Ryan Golant (ryan.golant@columbia.edu)\\
\textbf{Office:} Pupin 1333 (office hours by appointment)\\ 
\textbf{Time:} {Tuesdays, 6:00-9:00 PM} \\
\textbf{Location:} {Astronomy Library, Pupin 1402} \\

%%%%%%%%%%%%%%%%%%%%%%% OVERVIEW %%%%%%%%%%%%%%%%%%%%%%%
\section*{Class Overview}
Welcome to Astronomy Lab II! This lab class is designed to supplement the corresponding lecture courses \textit{Stars and Atoms} (UN1836) and \textit{Theories of the Universe: Babylon to the Big Bang} (UN1610). The objectives of this lab are for you to:
\begin{itemize}
\item Develop critical thinking skills and learn to apply scientific reasoning in your evaluation of information and arguments. 
\item Deepen your understanding of the Universe, of error and uncertainty in scientific measurements, and of quantitative problem solving.
\item Learn to frame and ask well-defined questions, to gather data and test hypotheses quantitatively, to model astronomical phenomena, and to communicate your findings in both written and verbal formats.
\end{itemize}
In the end, the purpose of this class is to allow you to explore the Universe in a hands-on manner and to ask questions in a more personal setting than that afforded by large lecture classes. My primary goal in teaching this class is for you to learn some cool new things about the vast expanse of Outer Space :) 

\bigskip
 
\noindent There will be 10 labs throughout the semester. Unless you need some extra time to finish a lab after class, there will be no homework assigned outside of our weekly 3-hour lab period. As a capstone project for this class, you will all give a short presentation on an astronomy topic of your choice.
 
%%%%%%%%%%%%%%%%%%%%%%% MATERIALS %%%%%%%%%%%%%%%%%%%%%%%
\section*{Lab Materials}
 
Please bring the following to each lab session:
 
\begin{itemize}
\item \textbf{A lab notebook:} This can be a bound notebook, but feel free to use an electronic document instead; if you choose the latter, make sure to have your device with you and ready to use.  
\item \textbf{Writing/drawing tools:} Pen, pencil, eraser, ruler, etc. Colored pens/pencils may come in handy but are not required.
\item \textbf{Scientific calculator:}  A calculator capable of performing trigonometric functions, logarithms, exponents, roots, etc. A graphing calculator is not required. You may find \url{https://www.desmos.com/scientific} or \url{https://www.wolframalpha.com/} useful if you wish to do calculations on your computer. 
\item \textbf{Laptop:} Laptops will be a necessity for many of the labs.  A limited number of laptops will be available for students who don't have their own. Some labs may require you to watch short videos online, so headphones or earbuds are also recommended. \\
\end{itemize}

%%%%%%%%%%%%%%%%%%%%%%% GRADING %%%%%%%%%%%%%%%%%%%%%%%
\section*{Grading}

\subsection*{Lab Write-ups}
Each lab will clearly denote what you should record in your write-ups for the lab. Lab responses can be recorded either in a bound physical notebook or in an electronic document. You may submit your work as a \textbf{PDF} to CourseWorks (strongly preferred), or you may hand in your lab notebook to me at the end of class, to be returned at the beginning of the following lab. All submissions will be due by midnight on the day of the lab; if this deadline poses a problem, please contact me.

\noindent While I strongly recommend you work through labs with a partner, each of you should keep your own records. The purpose of the lab write-ups is for you to explain to me \textit{what} you did during the lab, \textit{how} you did it, and \textit{why} you did it -- I'm much more concerned with the reasoning behind your arguments than I am with the format of your submission. When doing quantitative problem solving (in any discipline), always show your work and clearly explain your thought process. 

\subsection*{Participation}
\noindent Participation is an essential part of this lab. As science is a collaborative discipline, I recommend that you work through the labs with your peers, though you may work alone if you prefer. Your participation grade will be based on your contributions to your lab group, your class attendance, and your participation in class discussion. You are always encouraged to ask questions, regardless of content. You will also have the opportunity to record lingering questions in your lab write-ups.

\subsection*{Final Presentations}
\noindent For our final class session, you will each give a 10-minute presentation on a topic of your choice, followed by a 5-minute discussion with the class. A list of recommended topics related to astronomy and science in society will be provided, but you are also welcome to propose your own topics. Please talk to me if you would like to explore a topic not included on the list of recommended topics.

\subsection*{Grade Breakdown}
\noindent \textbf{75\%} Lab submissions*

\noindent \textbf{15\%} Final Presentations

\noindent \textbf{10\%} Participation

\noindent *Your lowest lab grade will be dropped when determining your final grade.

%%%%%%%%%%%%%%%%%%%%%%% SCHEDULE %%%%%%%%%%%%%%%%%%%%%%%
\section*{Tentative Schedule}

\begin{tabular}{cc}
    1/25 & (On Zoom) Lab 1: Orders of Magnitude\\
    2/01 & Lab 2: The Multiwavelength Universe\\
    2/08 & Lab 3: Spectroscopy\\
    2/15 & Lab 4: Galaxy Classification\\
    2/22 & Lab 5: The H-R Diagram\\
    3/01 & Lab 6: Dark Matter (w/ XENON guest speaker)\\
    3/08 & No Lab / Make-up Labs\\
    3/15 & Happy Spring Break!\\
    3/22 & Lab 7: Exoplanet Detection\\
    3/29 & Lab 8: Hubble's Law\\
    4/05 & Lab 9: Observational Astronomy\\
    4/12 & Lab 10: Numerical Methods\\
    4/19 & Presentation Prep / Make-up Labs\\
    4/26 & Final Presentations\\
    5/03 & No Class - Good Luck on Finals!!
\end{tabular}
\bigskip

Weather permitting, we may substitute one or two of these labs with an \emph{observing lab}, which would involve the use of the telescope on the roof of Pupin Hall. In this case, we may slightly shuffle the order of the remaining labs.

%%%%%%%%%%%%%%%%%%%%%%% POLICIES %%%%%%%%%%%%%%%%%%%%%%%
\section*{Policies}
 
\subsection*{Attendance}
 
By department policy, more than two unexcused (non-medical related) absences will result in automatic failure of the course. Please notify me if  extenuating circumstances arise (family emergencies, serious illness, quarantine requirement, or religious holidays) and we will arrange a make-up lab.
 
\subsection*{Accommodations}
If you have an identified disability, we encourage you to register with the Office of Disability Services: \url{https://health.columbia.edu/services/register-disability-services}
 
\subsection*{Academic Honesty}
\url{https://www.college.columbia.edu/academics/academicintegrity}
 
\subsection*{Mandatory reporting}
Instructors are required to report allegations of ``gender based misconduct, discrimination, or harassment" to Columbia's administration. While we are willing to listen and seek out resources (including confidential counselors) on your behalf, we cannot ourselves provide confidentiality.

\section*{Astronomy events at Columbia:}
Public lectures and observing sessions: \url{http://outreach.astro.columbia.edu/}

\end{document}
