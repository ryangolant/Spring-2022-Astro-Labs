\documentclass[11pt]{article}
\usepackage[includeheadfoot, top=1.0in, bottom=1.0in, hmargin=1.0in]{geometry}
\usepackage[utf8]{inputenc}
\usepackage{fancyhdr}
\usepackage{url}
\pagestyle{fancy}
\usepackage{setspace}
\usepackage{tabularx}
\usepackage{graphicx}
\usepackage{caption}
\usepackage{subcaption}
\usepackage{hyperref}
\usepackage{multicol}

\usepackage{hyperref}
\hypersetup{
    colorlinks=true,
    linkcolor=blue,
    filecolor=magenta,      
    urlcolor=blue,
}


\lhead{Astronomy Lab II}
\rhead{Spring 2022}
\lfoot{Golant}
\rfoot{Tues 6-9pm}
\cfoot{\thepage}

\begin{document}

\begin{center}
\huge{Lab 4: Galaxy Classification}\\ \medskip \Large{February 15, 2022}
\end{center}

%%%%%%%%%%%%%%%%%%%%%%% INTRO %%%%%%%%%%%%%%%%%%%%%%%
\section{Introduction: In a Galaxy Far, Far Away}
In the early 20th century, there was a strong tension amongst astronomers regarding the nature of some faint, fuzzy objects that could barely be resolved with a telescope. Some astronomers believed that these ``nebulae" were clusters of stars in our own Milky Way, while others believed that they were their own distant, independent islands of stars. This disagreement has since been deemed \textit{The Great Debate}. It wasn't until 1924 -- when Edwin Hubble measured the distance to one of these nebulae -- that astronomers were able to settle the debate: Hubble found the Andromeda ``nebula'' to be over \emph{2 million light years} away, confirming that this was indeed a \textbf{galaxy} far, far away (cue the \href{https://www.youtube.com/watch?v=MNMSAIG0dfQ}{Star Wars music}).


\medskip
Since Hubble's discovery, astronomers have studied galaxies with great interest. Galaxies come in all shapes, sizes, colors, and luminosities.  Each of these properties can tell us something unique about a galaxy: we can learn about the stars and gas within the galaxy, or the amount of dark matter permeating the galaxy; we can learn about the age of the galaxy and its evolutionary history; and we can learn about the galaxy's surrounding environment and the interactions that this environment induces. The first step in understanding a galaxy is to determine what \emph{type} of galaxy it is -- and we can do this by just looking at it!  Today, we'll think about the \textit{morphology} (or shape and structure) of galaxies and what we can learn from that. The goals of today's lab are for you to get a sense of the diversity of galaxies in the Universe, to create and test a galaxy classification system of your own, and to learn what the morphology of a galaxy can tell us about its evolutionary state.

%%%%%%%%%%%%%%%%%%%%%%% MAKE YOUR OWN CLASSIFICATION %%%%%%%%%%%%%%%%%%%%%%%
\section{Classifying Galaxies}
\subsection{Make Your Own Classification System}
Break up into groups of 2-3.  With your group, navigate to \url{http://cas.sdss.org/dr7/en/proj/advanced/galaxies/classification.asp}; you should see a table with 3 columns (``Run,'' ``Camcol,'' and ``Field'').  Clicking on a number in the ``Field'' column will bring you to an image of one or more galaxies.

\noindent
\textbf{Complete the following in your lab write-up}:
\begin{enumerate}
    \item Look through each of the ``Field'' images and spend a little time thinking about the similarities and differences between each galaxy. Make a list of some characteristics or qualities that you might use to describe these galaxies.
    
    \item Come up with a method of \emph{classifying} these galaxies -- that is, how might you sort these galaxies into different categories? Write out a series of steps or a visual representation of how your classification method works. (note: there is no ``correct'' answer to this question, but do keep in mind that all the images are at different distances, so the \emph{perceived} brightness and size of these galaxies will be different from their actual brightness and size)
    
    \item Once you've finished crafting your classification system, find the Hercules Cluster image in the ``Lab 4'' folder (under ``Files'') on CourseWorks. 
    \begin{enumerate}
        \item Using your classification method from the previous question, classify each labeled galaxy.
        
        \item How robust is your system? Comment on what faults your method may have (if any).
        
        \item Would your classification method still work well when applied to images at different wavelengths? (say, images of the same galaxy in X-rays or radio waves)
    \end{enumerate}  
\end{enumerate}
    
Let me know when you've finished up to this point. Once everyone's done, we'll get together as a class and discuss the different classification systems you came up with.
\medskip

%%%%%%%%%%%%%%%%%%%%%%% GALAXY ZOO %%%%%%%%%%%%%%%%%%%%%%%
\subsection{Galaxy Zoo}

Navigate to \url{https://www.zooniverse.org/projects/zookeeper/galaxy-zoo/}. ``Galaxy Zoo'' is a citizen science project that invites amateur astronomers to help categorize large data sets of galaxy images. Once you're on the Galaxy Zoo main page, click ``Get Started" and read through the instructions; if you need some help, you can click on ``Need some help with this task?" or pull out the ``Field Guide" to the right of the screen. For the first galaxy you look at, I highly recommend reading through some of the examples under ``Need some help with this task?" 

\medskip \noindent
Spend about 10-15 minutes classifying galaxies on Galaxy Zoo. \textbf{Complete the following in your lab write-up}:
\begin{enumerate}
\setcounter{enumi}{3}

    \item Make a table based on the galaxy classification system we agreed on as a class. Keep a tally of how many galaxies of each type show up on Galaxy Zoo.
    
    \item Once 10-15 minutes are up, briefly reflect on anything unusual you saw in the galaxies on Galaxy Zoo and how these factors might have complicated the classification system that we came up with together. How could we modify our classification system to accommodate for some of these features?
\end{enumerate}


%%%%%%%%%%%%%%%%%%%%%%% HUBBLE TUNING FORK %%%%%%%%%%%%%%%%%%%%%%%
\section{Cue the Music: The Hubble Tuning Fork}

\begin{figure} [h!]
    \centering
    \includegraphics[width=0.7\textwidth]{Images/The-Hubble-tuning-fork.png}
    \caption{The Hubble tuning fork.}
    \label{fig:Hubble}
\end{figure}

\noindent
Figure \ref{fig:Hubble} shows a simplified schematic of the galaxy classification system that we typically use in modern astronomy -- this diagram is known as the \textit{``Hubble tuning fork''} (or the ``Hubble classification system,'' or the ``Hubble sequence''). The Hubble sequence is a \emph{morphological} galaxy classification scheme introduced by Edwin Hubble in 1936 (the nickname of ``tuning fork'' comes from the shape in which this system is traditionally represented). Hubble's scheme divides galaxies into \emph{three} broad classes based on their visual appearance. \textbf{Answer the following in your lab write-up}:
\medskip
\begin{enumerate}
\setcounter{enumi}{5}
    \item There are three primary branches depicted on the tuning fork. 
    \begin{enumerate}
        \item What features distinguish each of the three branches from one another? (that is, how are the galaxies along one branch unique from the galaxies along the other two branches?)
        
        \item How do the galaxies \emph{within} each branch vary?
    \end{enumerate}
    
    \item How does the Hubble classification scheme compare with the galaxy classification system we came up with in class?
\end{enumerate}
\medskip

\noindent
You probably noted some of the following features in your description of the Hubble tuning fork; here are the formal names and descriptions for the galaxy types represented in Hubble's classification system:

\begin{itemize}
    \item \underline{Elliptical galaxies} (to the left of the tuning fork) appear as smooth, featureless ellipses in images. They are denoted by the letter `E,' followed by an integer `n' representing their degree of ``ellipticity'' (with $n=0$ being a spherical galaxy and $n=7$ being a very stretched-out ellipsoidal galaxy).
    
    \item \underline{Spiral galaxies} (to the right of the tuning fork) consist of a flattened ``disk'' with stars forming a spiral structure, along with a central concentration of stars known as the ``bulge,'' which is similar in appearance to an elliptical galaxy. Roughly half of all spirals are also observed to have a bar-like structure extending from the central bulge -- these ``barred spirals'' (on the bottom right of the tuning fork) are given the label `SB.' \emph{Unbarred} spiral galaxies (on the upper right of the tuning fork) are labeled with the letter `S.' 
    
    \item \underline{Lenticular galaxies} (at the center of the tuning fork) also consist of a bright central bulge surrounded by an extended disk-like structure -- but, unlike spiral galaxies, the disks of lenticular galaxies have no visible spiral structures and are not actively forming stars in any significant quantity. Lenticular galaxies are labeled `S0.'
    
    \item \underline{Irregular Galaxies} do not fit into the Hubble sequence, because they have no regular structure (neither disk-like nor ellipsoidal). Irregular galaxies are labeled `Irr.'
\end{itemize}


\noindent
\textbf{Here's a fun fact:} The Hubble tuning fork is drawn the way it is because astronomers originally thought that galaxies evolved from \emph{left} to \emph{right} along the diagram -- they thought that galaxies were formed as \textit{elliptical} galaxies and then evolved into \textit{spiral} galaxies, finally morphing into \textit{irregular} galaxies. We now think that galaxies do \emph{the exact opposite}, and are born as irregular galaxies, form spirals through interactions with other galaxies, and then finally end up as elliptical, featureless galaxies.
\medskip

\subsection{The Hubble Ultra Deep Field}
%Introduce the Hubble Deep Field

Navigate to \url{https://cdn.spacetelescope.org/archives/images/large/heic0406a.jpg}.
\\\textbf{\textit{Whoa}}, right? This image is known as the \textit{Hubble Ultra Deep Field}. The snapshot includes a myriad galaxies of various ages, sizes, shapes, and colors. The smallest, reddest galaxies may be among the most distant and oldest known, existing when the Universe was just 800 million years old (so, nearly \emph{14 billion} years ago). By contrast, the newest galaxy we know of formed only about 500 million years ago.

\medskip \noindent
The image required 800 exposures taken over the course of 400 orbits of the Hubble Space Telescope around Earth. The total exposure time was 11.3 days, taken between Sept. 24, 2003 and Jan. 16, 2004.

\medskip \noindent
\textbf{Complete the following in your lab write-up}:
\begin{enumerate}
\setcounter{enumi}{7}
    \item With the exception of a few stars in this image, every object you see is a galaxy. Try to identify the stars. How did you identify them?
    
    \item Estimate the total number of galaxies you see in the Hubble Ultra Deep Field. Note that this image covers only about \emph{one twenty-six-millionth} of the total area of the sky (!!)
    
    \item About what fraction of the Field are spiral galaxies? What fraction are elliptical galaxies?

\end{enumerate}

\medskip

%%%%%%%%%%%%%%%%%%%%%%% OBSERVING %%%%%%%%%%%%%%%%%%%%%%%
\section{Observing}

Now, we'll take a look at some real galaxies using our very own telescope! \textbf{Complete the following in your lab write-up}:
\begin{enumerate}
\setcounter{enumi}{10}
    \item For each galaxy we look at, briefly describe what you see. Try to classify the galaxies based on the Hubble tuning fork. If you're finding it difficult to classify a given galaxy, note this in your write-up and describe what's causing this difficulty. 
\end{enumerate}

%%%%%%%%%%%%%%%%%%%%%%% CONCLUSIONS %%%%%%%%%%%%%%%%%%%%%%%
\section{Wrapping Things Up}
\textbf{Answer the following questions in your lab write-up}:
\begin{enumerate}
\setcounter{enumi}{11}
    
    \item List some factors that might make classifying galaxies a difficult task.
    
    \item Briefly describe the Hubble tuning fork and how it functions as a galaxy classification system.
    
    \item Considering the wide variety of galaxies that you saw in the Hubble Ultra Deep Field. Beyond morphology (i.e., shape) what are some other features that we could use to classify galaxies?
    
    \item (Note: you don't need to record anything in your write-up for this question) There are many citizen science research projects like Galaxy Zoo that \emph{YOU} can participate in. Check out some of them here: \url{https://www.zooniverse.org/projects?discipline=astronomy&page=1&status=live}
    
    \item Write down at least one question that you still have after finishing this lab.
    
    \item If you have any feedback on how today's lab was run, or if you have any suggestions for future lab sessions, please let me know!
\end{enumerate}


\end{document}
