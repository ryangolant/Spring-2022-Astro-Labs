\documentclass[11pt]{article}
\usepackage[includeheadfoot, top=1.0in, bottom=1.0in, hmargin=1.0in]{geometry}
\usepackage[utf8]{inputenc}
\usepackage{fancyhdr}
\usepackage{url}
\pagestyle{fancy}
\usepackage{setspace}
\usepackage{tabularx}
\usepackage{graphicx}
\usepackage{caption}
\usepackage{subcaption}
\usepackage{hyperref}
\usepackage{multicol}
\usepackage{amsmath}
\usepackage{enumitem}

\usepackage{hyperref}
\hypersetup{
    colorlinks=true,
    linkcolor=blue,
    filecolor=magenta,      
    urlcolor=blue,
}


\lhead{Astronomy Lab II}
\rhead{Spring 2022}
\lfoot{Golant}
\rfoot{Tues 6-9pm}
\cfoot{\thepage}

\begin{document}

\begin{center}
\huge{Project Presentations}\\ \medskip \Large{April 26}
\end{center}

\section{Overview}
For our last lab (\textbf{April 26}), each of you will give a presentation on an astronomy topic of your choice.  I have provided a list of suggested topics here, but you are welcome to come up with a topic of your own.  All topics must be approved by me by Friday \textbf{April 15}; since everyone must present on a different topic, topics will be approved on a first-come, first-serve basis.  This is your opportunity to explore something that intrigues \textit{you} about astronomy and to share that with me and your classmates -- so, have fun with your presentations!  Class on April 19 will be set aside for presentation  preparation. I'm also available by email or by appointment if you have any questions or would like to practice your talk.

\section{Guidelines}
\subsection*{Preparation}
\begin{itemize}
    \item You should submit your presentation slides along with a list of resources used (e.g., research papers, popular science articles, websites, books, etc.) by midnight on April 25 (i.e., the night before the presentations). You should also include references in your presentation where appropriate; no special formatting or citation style is needed.
    \item Keep in mind that it is often more compelling to discuss one or a few specific topics rather than to gloss over a wide range of content.
\end{itemize}
\subsection*{Presentations}
\begin{itemize}
    \item Presentations should be less than \textbf{10 minutes in length}
    \item Each presentation will be followed by a \textbf{5 minute question period}
    \item You may use any combination of slides and/or whiteboard
    \item All listeners will be required to give feedback to the presenters. On the day of the presentations, printed feedback forms will be provided with the following questions:
    \begin{itemize}
        \item What's one thing you learned and/or enjoyed in this presentation?
        \item What's one strength of the presentation that aided clarity, engagement?
        \item If you were to give the same talk, what would you change to convey
            the ideas more clearly?
    \end{itemize}
    \item Come ready to ask questions during and after each talk; questions will count towards your participation grade. Any type of question is welcome (e.g., asking the presenter to clarify a statement, asking the presenter for more background information, asking the presenter hypothetical questions based on relevant scenarios) -- remember, there's no such thing as a bad questions! 
\end{itemize}

\subsection*{Grading}
This project will constitute 15\% of your final grade: 10\% for the presentation and 5\% for your participation. Here is a rubric\footnote{Chiefly adapted from the American Astronomical Society---Chambliss award rubric.} for your presentation:

\noindent
\textbf{Content: 70\%}
\begin{itemize}
%\itemsep0em
\item (35\%) Presenter introduces and describe(s) topic at level appropriate to this class [\underline{\hspace{5mm}}]
\item (40\%) Presenter explains extent of and limitations on our knowledge on the topic, including data/observations underlying knowledge [\underline{\hspace{5mm}}]
\item (20\%) Presenter provides context by drawing connections to, e.g., different areas of astronomy, concepts from lab or
lecture, other areas of science, areas outside of science, etc. [\underline{\hspace{5mm}}]
\item (5\%) Presenter chooses and cites appropriate references (i.e., goes beyond Wikipedia and popular press releases).  Presenter submits reference list. [\underline{\hspace{5mm}}]
\end{itemize} 

\noindent
\textbf{Delivery: 30\%}
\begin{itemize}
\item (35\%) Presentation has a logical flow that audience can follow [\underline{\hspace{5mm}}]
\item (25\%) Presenter can address reasonable audience questions [\underline{\hspace{5mm}}]
\item (20\%) Presentation aids (slides or board-work) are understood by audience [\underline{\hspace{5mm}}]
\item (10\%) Presenter stays within allotted time [\underline{\hspace{5mm}}]
\item (10\%) Presenter speaks clearly, and keeps the audience engaged (questions, activities, etc.) [\underline{\hspace{5mm}}]
\end{itemize}
{\small [\underline{\hspace{5mm}}] = easily and concisely (4), sufficiently (3),
is somewhat able to (2), barely to did not (1)
}
\pagebreak
\section{Suggested topics}

Please submit your proposed topics by midnight on April 15.

\medskip \noindent
A non-comprehensive list of suggested topics can be found below. You can choose something not listed, so long as it's within the realm of stars, galaxies, cosmology, or related topics relevant to our lab's focus area. It should be something you haven't covered in depth in class or in this lab.

\medskip \noindent
Remember, more focused/specific topics often yield more compelling presentations (and are often better suited for 10-minute presentations):

\medskip \noindent
\textbf{Good topic:} ``The Great Red Spot and other storms, vortices, and zonal flows on
Jupiter.''

\noindent
\textbf{Not-as-good topic:} ``Gas giant atmospheres.''

Here are some broad suggested topics; you will likely want to focus on a more specific aspect of your topic of choice.

\begin{itemize}[noitemsep]
    \item Galaxies (including our own)
        \begin{itemize}[noitemsep]
            \item Galactic dynamics (e.g., birth, growth, rotation of galaxies)
            \item Supermassive black holes
            \item Different theories of dark matter (or different dark matter candidates)
            \item The intergalactic medium (IGM)
            \item Dark matter halos and the dark matter content of different galaxies
            \item Dwarf galaxy satellites of the Milky Way
            \item Ultra-faint dwarf galaxies
            %\item Stellar life cycle---from birth to supernova!
            \item Dark energy
            \item Galaxy clusters
        \end{itemize}

    \item Stars (including our Sun)
        \begin{itemize}[noitemsep]
            \item Interior structure and chemistry of stars
            \item Asteroseismology or helioseismology 
            \item Stellar atmospheres or magnetospheres
            \item Stellar or solar winds
            \item The process of star formation (or the properties of star-forming regions in galaxies)
            %\item Stellar life cycle---from birth to supernova!
            \item Binary star systems
            \item Clusters of stars (globular clusters or open clusters)
            \item Specific types of star (e.g., T Tauri, RR Lyrae, Population III (the first stars))
        \end{itemize}

    \item (Exo)Planets
        \begin{itemize}[noitemsep]
            \item Solar system formation and history
            \item Proto-planetary disks
            \item Planet and planetesimal formation
            \item Brown dwarfs
            \item Exoplanet detection methods not discussed in class (e.g., microlensing, astrometry)
            \item Exoplanet atmospheres
        \end{itemize}
    
    \item Astrobiology
    \begin{itemize}[noitemsep]
        \item The Search for Extraterrestrial Life (SETI)
        \item The Drake equation
        \item Dyson spheres (or other hypothetical megastructures)
        \item Technosignatures vs. Biosignatures
        \item Communication and signal detection; candidate SETI signals
        \item Breakthrough Listen or Breakthrough Starshot
    \end{itemize}
    
    \item Telescopes and spacecrafts
        \begin{itemize}[noitemsep]
            \item Specific missions/projects (e.g., Hubble Space Telescope, James Webb Space Telescope, Kepler, TESS, Nancy Grace Roman Space Telescope, Vera C. Rubin Observatory, Thirty Meter Telescope).
            \item Astronomy at specific wavelengths (e.g., Radio astronomy and very-long-baseline interferometry (VLBI), sub-millimeter astronomy, X-ray astronomy, gamma-ray astronomy)
            \item NASA budget, missions, proposals (i.e., how funding decisions are made)
            \item Space policy (i.e., laws governing space)
        \end{itemize}

    \item Miscellaneous
        \begin{itemize}[noitemsep]
            \item The Big Bang and the early Universe (e.g., inflation, nucleosynthesis, the epoch of recombination, the epoch of reionization)
            \item The cosmic microwave background (CMB)
            \item Gravitational waves and LIGO
            \item Compact objects (Black holes, neutron stars, pulsars, magnetars, white dwarfs)
            \item High-energy explosions (Fast Radio Bursts or Gamma-Ray Bursts)
            \item A biographical presentation on a famous astronomer. If you do this, choose 1-2 scientific contributions to emphasize. Some suggestions for scientists:
            \begin{itemize}[noitemsep]
                \item Annie Jump Cannon (spectra of stars)
                \item Cecilia Payne-Gaposchkin (the composition of stars)
                \item Vera Rubin (dark matter)
                \item Jocelyn Bell Burnell (radio pulsars)
                \item Nancy Grace Roman (stellar classification and motion)
                \item Jill Tarter (SETI)
                \item Sara Seager (exoplanets)
                \item Caroline Herschel (comets)
                \item Annie Maunder (sunspots, solar corona, eclipses)
                \item Margaret Kivelson (solar wind, Europa’s ocean)
                
            \end{itemize}
            
            \item A recent or historically significant astronomy paper (I recommend searching through \url{https://ui.adsabs.harvard.edu/} or \url{https://arxiv.org/archive/astro-ph}, or asking me for help finding a paper).
            
            
            
        \end{itemize}
        
\end{itemize}

%Other options (equally encouraged):
%\begin{itemize}    
    
    %I recommend looking at the Daily Paper
%        Summaries on Astrobites (\url{https://astrobites.org/}).  This is a
%        blog that summarizes scientific papers at an introductory level;
%        summaries are written by astro graduate students and aimed for
%        undergrad/grad students alike.
%        Other sources of brief, accessible scientific papers include Nature
%        (\url{https://www.nature.com/}), Nature Astronomy
%        (\url{https://www.nature.com/natastron/}), and
%        Science (\url{https://www.sciencemag.org/}).
%        For popular press that can direct you to interesting papers, consider:
%        \url{https://www.quantamagazine.org/physics} or
%        \url{https://www.scientificamerican.com}.

%    \item Biographical study of a famous astronomer or planetary scientist.
%        If you do this, choose at least one scientific contribution to
%        emphasize.
%        \begin{itemize}[noitemsep]
%            \item Galileo, Kepler, etc.
%            \item Caroline Herschel (comets)
%            \item Annie Maunder (sunspots, solar corona, eclipses)
%            \item Annie Jump Cannon (spectra of stars)
%            \item Cecilia Payne-Gaposchkin (the composition of stars)
%            \item David Jewitt (trans-Neptunian objects, comets)
%            \item Margaret Kivelson (solar wind, Europa's ocean)
%            \item Carl Sagan (science communication; solar system,
%                astrobiology)
%            \item Jill Tarter (SETI)
%            \item Sara Seager (exoplanets)
%        \end{itemize}
%\end{itemize}

\end{document}



\section{Suggested Topics}

\end{document}